\documentclass[french]{article}
\usepackage[utf8]{inputenc}
\usepackage[a4paper, total={6in, 8in}]{geometry}
\usepackage{babel}
\usepackage{tocloft}
\usepackage{biblatex}
\addbibresource{biblio.bib}

\cftsetindents{section}{0em}{2em}
\cftsetindents{subsection}{0em}{2em}

\renewcommand\cfttoctitlefont{\hfill\Large\bfseries}
\renewcommand\cftaftertoctitle{\hfill\mbox{}}
\begin{document}

\begin{titlepage}
\newcommand{\HRule}{\rule{\linewidth}{0.5mm}}
\center
\textsc{\LARGE
Université de bordeaux
} \\[1cm]

\HRule \\[0.2cm]
{ \huge \bfseries Analyse des besoins \\[0.15cm] }
{ \bfseries IA pour un jeu de Capture the Flag temps réel en Python\\[0.15cm] }
\HRule \\[1.5cm]
Robin Navarro, Adrien Boitelle, Yann Blanchet\\Alexis Perignon, Alexis Flazinska
\\[1cm]
\today \\ [1cm]
\end{titlepage}

\newpage
\Large
\tableofcontents


\normalsize
\newpage
\section{Description du projet}

Ce projet a pour but la réalisation d'un framework pour un jeu de capture de drapeaux, et la réalisation d'IA pour y jouer.
Le jeu est un capture the flag à deux équipes, ayant chacune le même nombre de bot.
\newline

Chaque équipe évolue sur une carte symétrique dont elles ont connaissance dès le départ. Tous les membres d'une même équipe commençent dans une zone, la zone de départ, afin d'y ramener le drapeau de couleur adverse.
\newline

Chaque joueur dispose d'un champ de vision, qui lui permet de communiquer à son équipe des éléments comme par exemple la position d'un ennemi. Les joueurs peuvent attaquer les membres de l'équipe adverse afin de ralentir leur progression. Si un joueur meurt avec le drapeau, il le pose à terre.
\newline

Les IA devront piloter ces joueurs en temps réel, sur une carte en 2D, en implémentant des stratégies d'équipe.


\section{Analyse de l'existant}
    
Il y a un existant codé en Java pour le moteur physique mais la manière dont il est implémenté ne convient pas au client car il est difficile d'ajouter de nouvelles fonctionnalités. Ainsi, le client a débuté la conception d'un moteur de jeu en Python, mais ne l'a pas fini à ce jour. \newline
(Comme il était prévu que le jeu soit fonctionnel lorsque nous commençions le projet, nous pouvons en partie compter là dessus. Nous devons cependant prévoir la conception totale et partielle du moteur de jeu car dans tous les cas il y aura des additions à réaliser, notamment à l'aide de librairies externes.)

\section{Description des besoins}

\subsection{Besoins fonctionnels}
\subsubsection{Besoins utilisateur}
    \begin{itemize}
        \item Fournir une API pour que les IA communiquent avec le moteur de jeu. \\
                Priorité : Essentiel.\\

        \item Afficher une carte en deux dimensions. \\
            Priorité : Essentiel.
            \begin{itemize}
                \item Afficher les blocs de la map.
                \item Afficher les zones de chaque équipe.
                \item Afficher les objets dynamiques (bots, drapeau).
                \item Afficher le champ de vision des bots.
            \end{itemize}

    \end{itemize}

\subsubsection{Besoins système}
    
    \begin{itemize}

        \item Charger une carte en mémoire depuis un fichier texte. \\
                Priorité : Conditionnel.\\


        \item Les IA doivent gérer des équipes de 5 bots. \\
                Priorité : Essentiel.\\


        \item Gérer les collisions. \\
                Priorité : Essentiel.
                \begin{itemize}
                    \item Entre les bots et les éléments de la map.
                    \item Entre les projectiles des bots et les objets du jeu.\\
                \end{itemize}


        \item Déplacer les bots à l'aide d'un vecteur et d'une vitesse.\\
                Priorité : Essentiel.\\

                
        \item Détecter les objets dans le champ de vision des bots.
        \begin{itemize}
            \item Transmettre cette information aux ia.\\
        \end{itemize}
        
        \item Permettre aux bots de pouvoir tirer.
        \begin{itemize}
            \item Respecter la cadence de tir.
            \item Détecter lorsque qu'un bot est touché.
            \item Faire réaparaitre les bots une fois qu'ils sont touchés.
            \item Dire au bot qu'il s'est fait toucher.\\
        \end{itemize}


        \item Permettre au bot de récupérer le drapeau.
        \begin{itemize}
            \item Donner au bot le drapeau.
            \item Lacher le drapeau en cas de mort du bot.
            \item Victoire en cas de drapeau ramené dans la zone du bot.
        \end{itemize}
    \end{itemize}


\subsection{Besoins non fonctionnels}
\subsubsection{Besoins utilisateur}
    \begin{itemize}
        \item L'api doit être bien documentée car possiblement destinée à des étudiants.\\
            Quantificateur : Doit être compréhensible par un étudiant qui découvre le projet. \\

        \item L'affichage doit être fluide.\\
            Quantificateur : Le jeu doit pouvoir tourner à 30fps, en utilisant la bibliothèque pygame.\\
            Faisabilité : \\


    \end{itemize}
\subsubsection{Besoins système}
    \begin{itemize}

        \item Les IA doivent répondre rapidement.\\
            Quantificateur : Les réponses doivent être apportées à chaque frame.\\
            Faisabilité : \\
            Contrainte, difficulté technique : Le code étant en python, il est compliqué d'avoir des performances élevées.\\
            Risques et parades : Le risque est que l'IA ne réponde pas à temps. Dans ce cas, l'IA se fait disqualifier ou doit passer son tour.\\
    \end{itemize}

\subsection{Justifications}

\section{Diagramme de Gantt}

\section{Bibliographie}

% Les items de la bibliographie n'apparaissent que si ils sont cités
\printbibliography[title={}]

\end{document}